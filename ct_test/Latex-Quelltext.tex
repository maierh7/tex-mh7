\documentclass[fontsize=12pt]{scrartcl}

\usepackage[utf8]{inputenc}
\usepackage[T1]{fontenc}
\usepackage[ngerman]{babel}
\usepackage{graphicx}
\usepackage{booktabs}
\usepackage[round,authoryear]{natbib}

\title{Eine nobelpreisverdächtige Arbeit}
\author{Aus unbekannter Feder}

\begin{document}

\maketitle
\tableofcontents


\section{Einleitung}

\subsection{Stand der Forschung}

Während die traditionelle Latexproduktion bereits hinreichend erforscht ist (siehe Abbildung~\ref{fig:latex}), bleibt das wissenschaftliche Verständnis elektronischer Verarbeitungsprozesse dieses vielseitigen Materials weiterhin lückenhaft. 

\begin{figure}[b]
	\centering
	\includegraphics[width=3cm]{latex.jpg}
	\caption{Traditionelle Latexproduktion}
	\label{fig:latex}
\end{figure}

(siehe \citep[S.~25]{lamport1994latex}) 

\section{Methodik}

Unter Zuhilfenahme der Formeln~\ref{eq:ekin} und \ref{eq:impuls} werden wir diese Forschungslücke schließen. $E_\mathrm{kin}$ ist die kinetische Energie, $m$ die Masse und $\vec{v}$ die Geschwindigkeit.

\begin{equation}
	\label{eq:ekin}
	\sum E_\mathrm{kin} = \sum E'_\mathrm{kin}
\end{equation}

\begin{equation}
	\label{eq:impuls}
 	\vec{v_1} - \vec{v_1'} = \frac{m_2}{m_1} (\vec{v_2'} - \vec{v_2})
\end{equation}

\section{Ausblick}

Daraus ergeben sich gemäß Tabelle~\ref{tab:schritte} folgende nächste Schritte, deren sequenzielle Ausführung von essenzieller Bedeutung ist.

\begin{table}[h]
	\centering
	\caption{Nächste Schritte}
	\label{tab:schritte}
	\begin{tabular}{ll}
		\toprule
		\textbf{Nr.} & \textbf{Vorgehen} \\
		\midrule
		1 & Aktuellen Forschungsstand recherchieren \\
		2 & Methoden entwickeln \\
		3 & Schlussfolgerung aufstellen \\
		\bottomrule
	\end{tabular}
\end{table}

\bibliographystyle{plainnat}
\bibliography{literatur.bib}

\end{document}
\grid
\grid
\grid
\grid
\grid
\grid
\grid
\grid
