\documentclass[12pt,a5paper,landscape]{article}
\usepackage[a5paper,margin=0.5in]{geometry}
\usepackage[utf8]{inputenc}

\usepackage{amsmath} % für Mathematik
\usepackage{fancybox} % für shadowbox
\usepackage{rotating} % für turn-Umgebung

\RenewCommandCopy{\Hat}{\hat}
\usepackage{realhats} % für Hüte auf Buchstaben

% Für die kreative Darstellung von Pi
% Quelle: https://tex.stackexchange.com/questions/208426/how-do-i-display-pi-in-latex-like-don/208428#208428

\usepackage{scalerel}
\newlength\curht
\newlength\zshft
\newcounter{letcount}
\def\defaultdimfrac{.98}
\def\slantvalue{0}
\zshft=0pt\relax
\def\defaultstartht{\baselineskip}
\newcommand\diminish[2][\defaultdimfrac]{%
  \curht=\defaultstartht\relax
  \def\dimfrac{#1}%
  \setcounter{letcount}{0}
  \diminishhelpA{#2}%
}
\newcommand\diminishhelpA[1]{%
  \expandafter\diminishhelpB#1\relax%
}
\def\diminishhelpB#1#2\relax{%
  \raisebox{\value{letcount}\zshft}{\scaleto{\strut\slantbox{#1}}{\curht}}%
  \stepcounter{letcount}%
  \curht=\dimfrac\curht\relax%
  \ifx\relax#2\relax\else\diminishhelpA{#2}\fi%
}
\newsavebox{\foobox}
\newcommand{\slantbox}[2][\slantvalue]{\mbox{%
        \sbox{\foobox}{#2}%
        \hskip\wd\foobox
        \pdfsave
        \pdfsetmatrix{1 0 #1 1}%
        \llap{\usebox{\foobox}}%
        \pdfrestore
}}

\begin{document}

\thispagestyle{empty}
\large

\vspace{1cm}
\noindent
Manchmal \raisebox{2mm}{möchte} \raisebox{4mm}{Text} \raisebox{2mm}{wie}
eine \raisebox{-2mm}{Schlange} sein.

\vspace{-0.5cm}
\noindent
\begin{flushright}
    Dieser Test sieht ganz schön \begin{sideways} doof \end{sideways} aus.
\end{flushright}

\begin{center}
   \def\pinum{3.14159265358979323846264338327950288419716939937510}
    \def\slantvalue{.35}
    \zshft=.4pt\relax
    \def\defaultstartht{38pt}
    Das ist $\pi$: \diminish[0.92]{\pinum} 
\end{center}

\noindent
In der Mathematik haben Buchstaben manchmal \textbf{Hüte}. Doch nicht nur langweilige, wie diesen $\Hat{x}$. Manchmal sind die Buchstaben in Weihnachtsstimmung, $\hat[santa]{x}$, manchmal feiern sie Halloween, $\hat[witch]{x}$, und manchmal machen sie sich schick, $\hat[tophat]{x}$.

\vspace{1cm}
\noindent
\parbox[b]{15mm}{Hier purzelt eine kleine Box}
\begin{turn}{-60}
  \parbox[b]{15mm}{Hier purzelt eine kleine Box}
\end{turn}
\begin{turn}{-120}
  \parbox[b]{15mm}{Hier purzelt eine kleine Box}
\end{turn}
\begin{turn}{-180}
  \parbox[b]{15mm}{Hier purzelt eine kleine Box}
\end{turn}
\begin{turn}{-240}
  \parbox[b]{15mm}{Hier purzelt eine kleine Box}
\end{turn}
\begin{turn}{-300}
  \parbox[b]{15mm}{Hier purzelt eine kleine Box}
\end{turn}
\begin{turn}{-360}
  \parbox[b]{15mm}{Hier purzelt eine kleine Box}
\end{turn}
  
\end{document}