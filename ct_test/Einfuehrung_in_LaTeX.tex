\documentclass[12pt, a4paper]{article}
\usepackage[T1]{fontenc}
\usepackage[ngerman]{babel}

\title{Einführung in \LaTeX}
\author{Sabrina Patsch}
\date{\today}

\begin{document}

\maketitle

Hello World!

\begin{center}
    \LARGE Das Kaninchenloch hinab
\end{center}

Alice hatte es langsam \textbf{satt}, neben ihrer Schwester am Bachufer zu sitzen und nichts zu tun zu haben; sie hatte ein paarmal in das Buch geblickt, das ihre Schwester las, aber es waren {\Large keine Bilder oder Gespräche} darin, „und was für einen Sinn“, dachte Alice, „hat ein Buch \textsl{ohne Bilder oder Gespräche}?“

\begin{flushright}
    Gerade überlegte sie {\tiny (so gut sie konnte, denn die Hitze machte sie ganz schläfrig und dumm)}, ob das Vergnügen, eine Gänseblümchenkette zu flechten, die Mühe wert war, aufzustehen und die Gänseblümchen zu pflücken, als plötzlich ein weißes Kaninchen mit roten Augen dicht an ihr vorüber rannte.
\end{flushright}
\begin{center}
    \dots % drei Punkte
\end{center}

Das Kaninchenloch führte ein Stück geradeaus wie ein Tunnel und fiel dann plötzlich steil ab, so plötzlich,
dass Alice nicht einen Augenblick daran denken konnte anzuhalten, ehe sie einen anscheinend sehr tiefen
Schacht hinunterfiel. Und sie fiel eine Strecke von
\begin{equation}
    s(t) = \frac{1}{2} gt^2 + v_0 t + s_0
\end{equation}

\end{document}